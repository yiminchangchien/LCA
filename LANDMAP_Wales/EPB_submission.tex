\documentclass[]{elsarticle} %review=doublespace preprint=single 5p=2 column
%%% Begin My package additions %%%%%%%%%%%%%%%%%%%
\usepackage[hyphens]{url}

  \journal{Landscape and Urban Planning} % Sets Journal name


\usepackage{lineno} % add
\providecommand{\tightlist}{%
  \setlength{\itemsep}{0pt}\setlength{\parskip}{0pt}}

\usepackage{graphicx}
\usepackage{booktabs} % book-quality tables
%%%%%%%%%%%%%%%% end my additions to header

\usepackage[T1]{fontenc}
\usepackage{lmodern}
\usepackage{amssymb,amsmath}
\usepackage{ifxetex,ifluatex}
\usepackage{fixltx2e} % provides \textsubscript
% use upquote if available, for straight quotes in verbatim environments
\IfFileExists{upquote.sty}{\usepackage{upquote}}{}
\ifnum 0\ifxetex 1\fi\ifluatex 1\fi=0 % if pdftex
  \usepackage[utf8]{inputenc}
\else % if luatex or xelatex
  \usepackage{fontspec}
  \ifxetex
    \usepackage{xltxtra,xunicode}
  \fi
  \defaultfontfeatures{Mapping=tex-text,Scale=MatchLowercase}
  \newcommand{\euro}{€}
\fi
% use microtype if available
\IfFileExists{microtype.sty}{\usepackage{microtype}}{}
\bibliographystyle{elsarticle-harv}
\ifxetex
  \usepackage[setpagesize=false, % page size defined by xetex
              unicode=false, % unicode breaks when used with xetex
              xetex]{hyperref}
\else
  \usepackage[unicode=true]{hyperref}
\fi
\hypersetup{breaklinks=true,
            bookmarks=true,
            pdfauthor={},
            pdftitle={An hybrid approach of identifying outstanding scenic beauty areas by machine learning andscape designation},
            colorlinks=false,
            urlcolor=blue,
            linkcolor=magenta,
            pdfborder={0 0 0}}
\urlstyle{same}  % don't use monospace font for urls

\setcounter{secnumdepth}{0}
% Pandoc toggle for numbering sections (defaults to be off)
\setcounter{secnumdepth}{0}


% Pandoc header



\begin{document}
\begin{frontmatter}

  \title{An hybrid approach of identifying outstanding scenic beauty areas by
machine learning andscape designation}
    \author[School of Geography, University of Leeds]{Yi-Min Chang Chien\corref{1}}
   \ead{gyymcc@leeds.ac.uk} 
    \author[School of Geography, University of Leeds]{Alexis Comber}
   \ead{bob@example.com} 
    \author[School of Geography, University of Leeds]{Steve Carber\corref{2}}
   \ead{cat@example.com} 
      \address[Some Institute of Technology]{Department, Street, City, State, Zip}
    \address[Another University]{Department, Street, City, State, Zip}
      \cortext[1]{Corresponding Author}
    \cortext[2]{Equal contribution}
  
  \begin{abstract}
  This paper describes the potential of refining the process of landscape
  characterisation by incorporating the crowdsourced perceptions of scenic
  beauty into the widely adopted Landscape Character Assessment (LCA)
  approach in the UK. This study demonstrate the locally sensitive method
  widely used in the remote sensing studies for identifying the low
  concensus areas of scenic beauty between expert and non-expert
  perspectives within Wales.
  \end{abstract}
  
 \end{frontmatter}

\emph{Text based on elsarticle sample manuscript, see
\url{http://www.elsevier.com/author-schemas/latex-instructions\#elsarticle}}

\hypertarget{the-elsevier-article-class}{%
\section{The Elsevier article class}\label{the-elsevier-article-class}}

\hypertarget{introduction}{%
\paragraph{Introduction}\label{introduction}}

Landscape value is derived from human perception which is influenced by
personal feelings and opinions, it cannot be objective. Additionally,
when it comes to planning (e.g.~landscape designation), expert judgement
seems superior than public judgement. The single use of experts in
assessment methods can be practical while the resource is constrained.
characterisation of landscape using two dimensional information is
highly contested, therefore ignoring the importance of the general
public opinions. Expert judgement seems superior than public judgement.
--\textgreater{} how to measure whether there is consensus about the
value of a landscape or not

use visual and sensory aspect area as a framework for field survey

One of the practicality of the existing framework is the widely adopted
Landscape Character Assessment (LCA), which is based on two phases of
process: characterisation and evaluation. Firstly, landscape is
classified into homogenous categories in accordance with physical
character such as geology, landform, land cover. Secondly, However,
characterisation of landscape is a matter of interpretation and
perception and this task is typically reserved for professionals,
making--\textgreater{} designating criteria --\textgreater{} a few
expert judgement --\textgreater{} to some extent, landscape designation
is based on character --\textgreater{} Landscape Character Assessment
(descriptive v.s. evaluative) --\textgreater{} evaluative consideration
(i.e.~how is the evaluative components under current character system?)

Neogeography Scenic-Or-Not

Views of stakeholders and the public can be strong indicators as to
whether there is consensus about the value of a landscape

\begin{itemize}
\item
  Those involved in landscape planning were sometimes reluctant to
  tackle the visual and perceptual aspects of landscape, as opposed to
  the specific and often more easily dealt with aspects of land use and
  management, such as agriculture, forestry, recreation and nature
  conservation.
\item
  Scenic-Or-Not provides an invaluable data source which can be
  manipulated and interro- gated in many different ways and linked to
  other data sets.
\item
  In order to make these judgements (some of which are subjective) in a
  transparent and consistent way, this Guidance sets out which criteria
  Natural England intends to use.
\item
  It is certain in the Guidance for AONB designation that judgement has
  to be made in order to value the landscape;
\item
  No information is provided on how to involve stakeholders and how to
  measure whether there is consensus about the value of the landscape or
  not;
\item
  The single use of experts in assessment methods is highly contested in
  landscape preference research;
\item
  The practicality of the existing framework is questionable.
\item
  Nevertheless, questions need to be raised here. First of all, who is
  describing value to the landscape? Is the designation based on public
  judgment, or on expert judgement? This is not being stressed in the
  document, only it states that: `where there is a consensus of opinion
  that an area meets the statutory criteria or should be designated,
  this helps determining whether it is accorded a special value that
  should be recognized. Views of stakeholders and the public can be
  strong indicators as to whether there is consensus about the value of
  a landscape'. Unfortunately, no information is provided on how to
  involve stakeholders and how to measure whether there is consensus
  about the value of the landscape or not.
\item
  Characterisation of landscape is a matter of interpretation and
  perception;
\item
  the landscapes are classified on classified of low, moderate, high or
  outstanding quality. \# qualitative to quantitative evaluation The
  current practice of LCA is based on qualitative descriptions. The
  evaluation is carried out by specialists and recorded in a qualitative
  manner where the evaluative components under current character system
  includes wildness/integrity/ The relationship between wildness and
  scenicness was evidence by our previous attempt (forthcoming). Thus,
  it might be worth to look at whether the qualitative measure of
  wildness quality can be enhanced the qualitative evaluation of scenic
  quality. the sources involved are not quantitative: they are texts
  containing qualitative descriptions and, as such, need to be analysed
  using a mix of approaches that combine spatial analysis with close
  reading.
\item
  Some of the criteria used to value the landscape are also used as
  landscape character indica-tors for perceptual aspects such as
  tranquillity and wildness.  However, it is not stated how to use
  these criteria in practice or what to do with these indicators in the
  first place. It is merely mentioned that they can be used to identify
  valued landscapes that merit some form of designa-tion or recognition.
   Yet, in what way are the stakeholders involved? How can we derive
  the value that stakeholders assign to the landscape from these
  criteria? This does not seem to be explained in the LCA document. \#
  sentence
\item
  the point on the east bank of Windermere, for instance, gives little
  indication of precisely to where on this large body of water the text
  is referring
\item
  1:100,000, National land use policy: deciding which areas to protect
\item
  1: 25,000, Monitoring regional land use Local management: Managing
  sensitive areas
\item
  lt must be possible for decision-making bodies to update the
  information on their areas easily and quickly.
\item
  budgetary constraints and time limits.
\item
  The maps can be fairly easily updated on a regular basis.
\item
  The results of certain statistical surveys (areolar surveys of land
  use/cover) might suggest that the problem of the unit area is not
  really problematical. However, it must be remembered that these
  surveys sample points, not surfaces. Nonetheless, in the field it is
  still possible to classify a point as an item in a land cover
  nomenclature
\item
  it must provide an acceptable representation of reality.
\item
\item
\item
  (Seresinhe et al., \protect\hyperlink{ref-Seresinhe2019}{2019})
  \#\#\#\# 3. Materials and methods The process of landscape
  characterisation in the UK is primarily based on two-dimensional maps.
  The scenic quality is evaluated by professionals through site survey.
\end{itemize}

LANDMAP is Welsh national landscape landscapes --\textgreater{} The
planning authorities intend to use LANDMAP to inform decision making.
\#\#\# 3.1. Welsh LANDMAP Data LANDMAP is

The expert-based landscape qualities LANDMAP is an all-Wales landscape
resource where landscape characteristics, qualities and influences on
the landscape are recorded and evaluated in a nationally consistent data
set.

Characterisation sub-divides the landscape into areas based on the
visual continuity of physical characteristics such as geology, landform
and land cover applied through the lens of spatial hierarchical mapping.

The Visual \& Sensory aspect is organised according to a hierarchical
classification system. This typology aims to classify the landscape into
areas of distinct Visual \& Sensory character, and is based on a
hierarchy of four levels.

five aspects of landscape
classification/characteristics/characterisations --- geology, habitats,
sensory (primarily visually), historic and cultural characters. These
are evaluated and recorded as spatial data.

\begin{itemize}
\tightlist
\item
  geological landscape (classification)
\item
  landscape habitats (classification
\item
  visual \& sensory (classification): maps landscape characteristics and
  qualities as perceived through our senses, primarily visually.
  --\textgreater{} The physical attributes of landform and land cover,
  their visible patterns and their interrelationship. --\textgreater{}
  Examples of information recorded includes: Landscape elements and
  features, aesthetic presence of water and seasonality, important
  qualities, boundary type, scale, enclosure, diversity, colour, sense
  of place, night time light, building materials, views, scenic quality,
  rarity, integrity and character. • built land • coastal • coastal
  waters • developed unbuilt land • exposed upland/plateau • flat
  lowland/levels • hills, lower plateau \& scarp slopes • inland water
  (including associated edge) • lowland valleys • rolling lowland •
  upland valleys
\item
  historic landscape
\item
  LANDMAP is managed as a spatial dataset • Spatial data consists of
  observations (surveys in LANDMAP) with locations (mapped geographic
  areas) • Geographical Information System (GIS) software records and
  displays this spatial data
\end{itemize}

Maps and classifies landscapes -- Describes their characteristics,
qualities and components

The Welsh landscape baseline LANDMAP data : Visual and Sensory aspect

\hypertarget{scenic-or-not-data}{%
\paragraph{3.2. Scenic-Or-Not data}\label{scenic-or-not-data}}

is on 1-10 scores. The crowdsourced scenic ratings were reclassified
into one of scenic quality classes to correspond to the LANDMAP dataset.

\hypertarget{analysis}{%
\subsection{Analysis}\label{analysis}}

\hypertarget{oridnal-regression-model}{%
\paragraph{Oridnal Regression Model}\label{oridnal-regression-model}}

Once the package is properly installed, you can use the document class
\emph{elsarticle} to create a manuscript. Please make sure that your
manuscript follows the guidelines in the Guide for Authors of the
relevant journal. It is not necessary to typeset your manuscript in
exactly the same way as an article, unless you are submitting to a
camera-ready copy (CRC) journal.

\hypertarget{analysis-1}{%
\paragraph{Analysis}\label{analysis-1}}

apply the model to a case study area and suggest a character area by
grouping adjacent pixels with the same class. I suppose this would have
to be on a 1km grid for the Scenicness data

\begin{itemize}
\item
  document style
\item
  baselineskip
\item
  front matter
\item
  keywords and MSC codes
\item
  theorems, definitions and proofs
\item
  lables of enumerations
\item
  citation style and labeling.
\end{itemize}

\hypertarget{front-matter}{%
\section{Front matter}\label{front-matter}}

The author names and affiliations could be formatted in two ways:

\begin{enumerate}
\def\labelenumi{(\arabic{enumi})}
\item
  Group the authors per affiliation.
\item
  Use footnotes to indicate the affiliations.
\end{enumerate}

See the front matter of this document for examples. You are recommended
to conform your choice to the journal you are submitting to.

\hypertarget{bibliography-styles}{%
\section{Bibliography styles}\label{bibliography-styles}}

There are various bibliography styles available. You can select the
style of your choice in the preamble of this document. These styles are
Elsevier styles based on standard styles like Harvard and Vancouver.
Please use BibTeX~to generate your bibliography and include DOIs
whenever available.

Here are two sample references: ({\textbf{???}}; {\textbf{???}}).

\hypertarget{references}{%
\section*{References}\label{references}}
\addcontentsline{toc}{section}{References}

\hypertarget{refs}{}
\leavevmode\hypertarget{ref-Seresinhe2019}{}%
Seresinhe, C.I., Preis, T., MacKerron, G., Moat, H.S., 2019. Happiness
is Greater in More Scenic Locations. Scientific Reports 9, 4498.
doi:\href{https://doi.org/10.1038/s41598-019-40854-6}{10.1038/s41598-019-40854-6}


\end{document}


